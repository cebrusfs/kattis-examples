% %%%%%%%%%%%%%%%%%%%%%%%%%%%%%%%%%%%%%%%%%%%%%%%%%%%%%%%%%%%%%%%%%%%%%%%%%%%%%%
% LaTeX Template (Beamer)
%   Golbal Settings
%
% Author:
%   MengHuan Yu     cebrusfs@gmail.com
%                                                            2014 / 09 / 28
% %%%%%%%%%%%%%%%%%%%%%%%%%%%%%%%%%%%%%%%%%%%%%%%%%%%%%%%%%%%%%%%%%%%%%%%%%%%%%%

%\documentclass[usenames, dvipsnames]{beamer}
% ------------------------------------------------------------------------------
%   Style
% ------------------------------------------------------------------------------
\setbeamertemplate{navigation symbols}{} % 取消工具列
\useoutertheme{infolines}                % 開啟狀態列

\usetheme{CambridgeUS}
\usecolortheme{orchid}
\usefonttheme{professionalfonts}


% ------------------------------------------------------------------------------
%    邊界
% ------------------------------------------------------------------------------
%\setlength{\parskip}{8pt plus 1pt minus 1pt}


% ------------------------------------------------------------------------------
%    數學
% ------------------------------------------------------------------------------
\usepackage{mathtools}
\usepackage{amssymb, mathrsfs}

\DeclareMathOperator*{\Span}{span}        % span symbol
\DeclareMathOperator*{\Tr}{tr}            % trace symbol
\DeclareMathOperator*{\sign}{sign}        % sign symbol

\DeclarePairedDelimiter{\abs}{\lvert}{\rvert}
\DeclarePairedDelimiter{\norm}{\lVert}{\rVert}
\DeclarePairedDelimiter{\ceil}{\lceil}{\rceil}
\DeclarePairedDelimiter{\floor}{\lfloor}{\rfloor}

\DeclareMathOperator{\E}{\mathrm{E}}
\DeclareMathOperator{\Var}{\mathrm{Var}}
\DeclareMathOperator{\Cov}{\mathrm{Cov}}

\DeclareMathOperator*{\argmin}{arg\,min }
\DeclareMathOperator*{\argmax}{arg\,max }

% ------------------------------------------------------------------------------
%    程式碼
% ------------------------------------------------------------------------------
\usepackage{listings}

\lstset{%
    language        = {c++},
    % Font Setting
    basicstyle      = {\linespread{1.1}\ttfamily\small},
    identifierstyle = {\color{black}},
    commentstyle    = {\usefont{T1}{pcr}{b}{sl}\color{commentgreen}},
    keywordstyle    = {\color{blue}},
    stringstyle     = {\color{purple}},
    directivestyle  = {\color{brown}},
    % Single frame around code
    frame           = {single},
    % Align Setting
    xleftmargin     = {2em},
    xrightmargin    = {2em},
    % Line number Setting
    numberstyle     = {\tiny}, % Line numbers are blue and small
    numbers         = {left}, % Line numbers on left
    firstnumber     = {1}, % Line numbers start with line 1
    stepnumber      = {1}, % Line numbers go in steps of 1
    % Auto break line if code is too long
    breaklines      = {true},
    breakautoindent = {true},
    % Other
    showstringspaces= {false}, % Don't put marks in string spaces
    tabsize         = {4}, % 4 spaces per tab
    escapeinside    = {{/*}{*/}}, % can use for Chinese
    extendedchars   = {false},
    morecomment     = {[l][]{...}}, % Line continuation (...) is comment
}

% usage: \insertcode{language}{code path}{label}
%   ex. \insertcode{C++}{hw2_3_3.cpp}{Algorithm of 2-3}
\newcommand{\insertcode}[3]
{%
    \begin{itemize}
        \item[]\lstinputlisting[language=#1,caption=#3,label=#2]{#2}
    \end{itemize}
}


% ------------------------------------------------------------------------------
%    虛擬碼
% ------------------------------------------------------------------------------
%\usepackage[lined, boxed, commentsnumbered, ruled]{algorithm2e}


% ------------------------------------------------------------------------------
%    圖片
% ------------------------------------------------------------------------------
\usepackage{subfigure}
\usepackage{graphicx}

%includegraphics[height=高度, width=寬度, angle=旋轉角, scale=縮放倍數]%{檔名}

%\begin{figure}[here]
    %\centering
    %\setlength{\fboxrule}{0.5pt}
    %\setlength{\fboxsep}{0.3cm}
    %\fbox{\includegraphics[scale=1.0]{./1.png}}
    %\caption{comment}
%\end{figure}

% ------------------------------------------------------------------------------
%    xeCJK
% ------------------------------------------------------------------------------
\usepackage{indentfirst} % 中文第一段縮排套件
\usepackage[no-math]{fontspec}
\usepackage[SlantFont]{xeCJK}

\newcommand{\FontPath}{../../gen_pdf/font/}

\XeTeXinputencoding "utf8"
\XeTeXdefaultencoding "utf8"
\XeTeXlinebreaklocale "zh"
\XeTeXlinebreakskip = 0pt plus 1pt

\setmainfont[
    Mapping = tex-text,
    Path = \FontPath,
    UprightFont = *-Regular,
    BoldFont = *-Bold,
    BoldItalicFont = *-BoldItalic,
    ItalicFont = *-Italic,
]{NotoSerif}

\setsansfont[
    Mapping = tex-text,
    Path = \FontPath,
    UprightFont = *-Regular,
    BoldFont = *-Bold,
    BoldItalicFont = *-BoldItalic,
    ItalicFont = *-Italic,
]{NotoSans}

\setmonofont[
    Scale = 0.96,
    Path = \FontPath,
    UprightFont = *,
    BoldFont = *-Bold,
    BoldItalicFont = *-BoldItalic,
    ItalicFont = *-Italic,
]{Inconsolata-LGC}

\setCJKmainfont[
    Mapping = tex-text,
    Path = \FontPath,
    UprightFont = *-Regular,
    BoldFont = *-Bold,
]{NotoSansCJKtc}

\setCJKmonofont[
    Mapping = tex-text,
    Path = \FontPath,
    UprightFont = *-Regular,
    BoldFont = *-Bold,
]{NotoSansCJKtc}



% ------------------------------------------------------------------------------
%    繪圖
% ------------------------------------------------------------------------------
\usepackage{tikz}


% ------------------------------------------------------------------------------
%    雜七雜八
% ------------------------------------------------------------------------------
\usepackage{comment}         %latex註解插件

% ------------------------------------------------------------------------------
%   Section header
% ------------------------------------------------------------------------------
%\AtBeginSection[]
%{%
%    \begin{frame}
%        \tableofcontents[currentsection]
%    \end{frame}
%}
%\AtBeginSubsection[]
%{%
%    \begin{frame}[shrink]
%        \tableofcontents[sectionstyle=show/shaded,subsectionstyle=show/shaded/hide]
%    \end{frame}
%}

% ------------------------------------------------------------------------------
%   Auto framebreaks
% ------------------------------------------------------------------------------
\let\oldframe\frame
\renewcommand\frame[1][allowframebreaks, containsverbatim]{\oldframe[#1]}
\setbeamertemplate{frametitle continuation}[from second]

% ------------------------------------------------------------------------------
%   End of Header
% ------------------------------------------------------------------------------
