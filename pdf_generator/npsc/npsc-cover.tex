\begin{center}
	\LARGE \bf
    \NPSCYear \ContestName\\
    \NPSCGroup\NPSCContest
\end{center}


\begin{itemize}
    \item 本次比賽共 \NPSCProblemAmount\ 題,含本封面共 \pageref{LastPage} 頁。

    \item 全部題目的輸入都來自{\bf 標準輸入}。輸入中可能包含多組輸入,以題目敘述為主。

    \item 全部題目的輸出皆輸出到螢幕({\bf 標準輸出})。\\
          輸出和裁判的答案必須完全一致,英文字母大小寫不同或有多餘字元皆視為答題錯誤。

    \item 每一題的執行時間限制,請參考 Kattis 上的題目敘述。

    \item 比賽中上傳之程式碼,使用 C 語言請用 \texttt{.c} 為副檔名;使用 C++ 語言則用 \texttt{.cpp} 為副檔名。

    \item 使用 \texttt{cin} 輸入速度遠慢於 \texttt{scanf} 輸入,若使用需自行承擔 Time Limit Exceeded 的風險。

    \item 使用 \texttt{scanf} 或 \texttt{printf} 處理長整數 (\texttt{long long int})時,請使用 ``\texttt{\%lld}''。\\
            詳細可參閱下頁之輸入輸出範例。

    \item 部分題目有浮點數輸出,會採容許部分誤差的方式進行評測。一般來說「相對或絕對誤差小於 $\epsilon$ 皆視為正確」,$\epsilon$ 值以題目敘述為主。

        舉例來說,假設 $\epsilon = 10^{-6}$ 且 $a$ 是正確答案,$b$ 是你的答案,如果符合 $\frac{\abs{a - b}}{\max(a, b, 1)} \leq 10^{-6}$,就會被評測程式視為正確。

\end{itemize}



\clearpage

\begin{center}
    \LARGE \bf
    \NPSCYear \ContestName\\
    輸入輸出範例
\end{center}


C 程式範例:
\begin{lstlisting}[frame=single]
#include <stdio.h>
int main()
{
    int cases;
    scanf("%d", &cases);
    for (int i = 0; i < cases; ++i)
    {
        long long a, b;
        scanf("%lld %lld", &a, &b);
        printf("%lld\n", a + b);
    }
    return 0;
}
\end{lstlisting}

C++ 程式範例:
\begin{lstlisting}[frame=single]
#include <iostream>
int main()
{
    int cases;
    std::cin >> cases;
    for (int i = 0; i < cases; ++i)
    {
        long long a, b;
        std::cin >> a >> b;
        std::cout << a + b << std::endl;
    }
    return 0;
}
\end{lstlisting}

\clearpage

\clearpage
